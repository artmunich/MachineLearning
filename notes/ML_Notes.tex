\documentclass{article}
\usepackage{amsmath}
\usepackage{graphicx}
\usepackage{hyperref}

\begin{document}

%---week 1----------------------
\section{Week 1}

\subsection{Model Representation}
	\begin{displaymath}
		h_\theta(x)=\theta_0+\theta_1x+\epsilon
	\end{displaymath}
	
\subsection{Cost Function}
	
	$$J(\theta_0,\theta_1)=\frac{1}{2m}\sum_{i=1}^{m}(h_\theta(x^{(i)})-y^{(i)})^2$$

\subsection{Gradient Descent}
	Want: $min J(\theta_0,\theta_1)$
	
	Algorithm:
	repeat until convergence \{
	
	 $\theta_j := \theta_j - \alpha\frac{\partial}{\partial\theta_j}J(\theta_0,\theta_1),j=0,1$
	 
	 simutaneously update $\theta_j$
	 
	  \}

%-------------week 2----------------------------
\section{Week 2}
\subsection{Multivariate Linear Regression Model}
\textbf{Hypothesis:}
$$h_\theta(x)=X\Theta$$
where
\begin{equation*}
X=
\left(\begin{array}{ccccc}
1 & x^{(1)}_1 & x^{(1)}_2 & \cdots & x^{(1)}_n \\
1 & x^{(2)}_1 & x^{(2)}_2 & \cdots & x^{(2)}_n \\
\vdots & \vdots & \vdots &          &\vdots \\
1 & x^{(m)}_1 & x^{(m)}_2 & \cdots & x^{(m)}_n \\
\end{array}
\right)_{m\times{(n+1)}},
\Theta =
\left(\begin{array}{c}
\theta_0\\
\theta_1\\
\vdots\\
\theta_n
\end{array}
\right)_{(n+1)\times 1}
\end{equation*}

\subsection{Cost function(Vectorization)}
$$J(\theta)=\frac{1}{2m}(X\Theta-\vec{y})^T(X\Theta-\vec{y})$$

\subsection{Normal Equation}
$$\Theta = (X^TX)^{-1}X^T\vec{y}$$
\textbf{Deduction:}
	\begin{eqnarray}
		J(\theta)=\frac{1}{2m}(X\Theta-\vec{y})^T(X\Theta-\vec{y})\nonumber\\
			=\frac{1}{2m}\lVert X\Theta - \vec{y}\rVert^2 \nonumber\\
		\nabla J(\Theta)=\frac{1}{m}X^T(X\Theta-\vec{y})\nonumber\\
		\nabla J(\Theta)=0 \nonumber \\
		X^TX\Theta-X^T\vec{y}=0 \nonumber \\
		\Theta = (X^TX)^{-1}X^T\vec{y} \nonumber
	\end{eqnarray}	
	
%-----------------week 3------------------
\section{Week 3}

\subsection{Classification} %----3.1-----
\flushleft
Classification: $y\in \{0,1\}$, $h_\theta(x) $ can be $> 1$ or  $< 0$.

Logistic Regression: $0\leq h_\theta(x)\leq 1$

\subsection{Logistic Regression} %-----3.2-----
\textbf{Sigmoid(logistic) function}
\begin{displaymath}
	g(t)=\frac{1}{1+e^{-t}}
\end{displaymath}

\textbf{Logistic Regression Model:}
	\begin{displaymath}
		h_\theta(x)=\frac{1}{1+e^{-\Theta^Tx}}
	\end{displaymath}
\textbf{Interpretation:}
	\begin{itemize}
		\item $h_\theta(x)=$estimated probability that $y=1$ on input $x$.
		\item $h_\theta(x)=P(y=1|x;\theta)$ means probability that $y=1$, given $x$, parameterized by $\theta$.
		\item $P(y=0|x;\theta)+P(y=1|x;\theta)=1$	
	\end{itemize}

\subsection{Decision Boundary} %----3.3------------

Descion boundary is a property of the hypothesis and parameters.

\textbf{Linear decision boundary}
$h_\theta(x)=g(\theta_0+\theta_1*x_1+\theta_2*x_2)$:

$$\theta_0+\theta_1x_1+\cdots+\theta_nx_n=\sum_{i=0}^{n}=\Theta^Tx$$

\textbf{Non-linear decision boundary}

\subsection{Cost Function}
$$	J(\theta)=\frac{1}{m}\sum_{i=1}^{m}Cost(h_\theta(x^{(i)}),y^{(i)})$$
\begin{displaymath}
		Cost(h_\theta(x),y)=\left\{\begin{array}{ll}
					-\log(h_\theta(x)) & y=1\\
					-\log(1-h_\theta(x)) & y=0
					\end{array}\right.
\end{displaymath}
$h_\theta(x)$ is hypothesis, $y$ is training examples.

\textbf{Simplified cost function}
$$J(\theta)=-\frac{1}{m}[\sum_{i=1}^{m}y^i\log h_\theta(x^i)+(1-y^i)\log (1-h_\theta(x^i))]$$

\subsection{Gradient Descent}
\textbf{To fit parameters $\theta$}:
$\min J(\theta)$

\textbf{To make a prediction given new $x$}:

output $p(y=1|x;\theta)=h_\theta(x)=\frac{1}{1+e^{-\Theta^Tx}}$

\textbf{Gradient Descent:}
Repeat \{
\begin{eqnarray}
\theta_j := \theta_j - \alpha\frac{\partial}{\partial\theta_j}J(\theta) \nonumber \\
:= \theta_j - \alpha\sum_{i=1}^{m}(h_\theta(x^{(i)})-y^{(i)})x_j^{(i)},(j=0\cdots n)\nonumber
\end{eqnarray}
\}

Caution: Simulataneously update all $\theta_j$.

Click \href{http://blog.csdn.net/dongtingzhizi/article/details/15962797}{here for detailed deduction}.

\subsection{Advanced Optimization}
[x,fval,exitflag]=fminunc(fun,x0,options)

\subsection{Multiclass Classification}
\textbf{one-vs-all}

\subsection{Regularization}
\textbf{Underfitting, just right, overfitting}

Addressiong overfitting:
\begin{enumerate}
	\item Reduce number of features:
		\begin{itemize}
			\item Manually select which features to keep;
			\item Model selection algorithm
		\end{itemize}
	\item Regularization:
		\begin{itemize}
			\item Keep all features, but reduce magnitude/values of parameters $\theta_j$;
			\item Works well when there are a lot of features. 
		\end{itemize}
\end{enumerate}

\textbf{Regulariztion:}

Small values for parameters $\theta_0,\theta_1,\cdots,\theta_n$
\begin{itemize}
	\item "Simpler" hypothesis
	\item Less prone to overfitting.
\end{itemize}
\textbf{Cost function}
$$J(\theta)=\frac{1}{2m}[\sum_{i=1}^{n}(h_\theta(x^{(i)})-y^{(i)})^2 + \lambda\sum_{j=1}^{n}\theta_j^2]$$
$$\min J(\theta)$$

\subsection{Regularized linear/logistic regression}
\textbf{Cost function and gradient}
$$J(\theta)=\frac{1}{m}\sum_{i=1}^{m}[-y^i\log h_\theta(x^{(i)})-(1-y^{(i)})\log (1-h_\theta(x^{(i)}))]+\frac{\lambda}{2m}\sum_{j=1}^{n}\theta_j^2$$

\subsection{Vectorization}
\begin{equation*}
X=
\left(\begin{array}{ccccc}
x^{(1)}_0 & x^{(1)}_1 & x^{(1)}_2 & \cdots & x^{(1)}_n \\
x^{(2)}_0 & x^{(2)}_1 & x^{(2)}_2 & \cdots & x^{(2)}_n \\
\vdots & \vdots & \vdots &          &\vdots \\
x^{(m)}_0 & x^{(m)}_1 & x^{(m)}_2 & \cdots & x^{(m)}_n \\
\end{array}
\right)_{m\times{(n+1)}},
\Theta =
\left(\begin{array}{c}
\theta_0\\
\theta_1\\
\vdots\\
\theta_n
\end{array}
\right)_{(n+1)\times 1}
y =
\left(\begin{array}{c}
y^{(1)}\\
y^{(2)}\\
\vdots\\
y^{(m)}
\end{array}
\right)_{m\times 1}
\end{equation*}

So,

\begin{displaymath}
A=X\Theta=
\left(\begin{array}{c}
x^{(1)}_0\theta_0 + x^{(1)}_1 \theta_1 + x^{(1)}_2\theta_2 + \cdots + x^{(1)}_n\theta_n \\
x^{(2)}_0\theta_0 + x^{(2)}_1\theta_1 + x^{(2)}_2\theta_2 + \cdots + x^{(2)}_n\theta_n \\
\vdots  \\
x^{(m)}_0\theta_0 + x^{(m)}_1\theta_1 + x^{(m)}_2\theta_2 + \cdots + x^{(m)}_n\theta_n \\
\end{array}
\right)_{m\times 1}
\end{displaymath}

We have,

\begin{displaymath}
E=h_\theta(x)-y=g(X\Theta)-y=
\left(\begin{array}{c}
g(A^{(1)})-y^{(1)}\\
g(A^{(m)})-y^{(m)}\\
\vdots\\
g(A^{(m)})-y^{(m)}
\end{array}
\right)_{m\times 1}
\end{displaymath}.

Now let's take a look at the update of $\theta$:
\begin{eqnarray}
\theta_j := \theta_j - \alpha*\sum_{i=1}^{m}(h_\theta(x^{(i)})-y^{(i)})x_j^{(i)} \nonumber \\
:= \theta_j - \alpha * \sum_{i=1}^{m}E^{(i)}*x_j^{(i)} \nonumber\\
:= \theta_j - \alpha * X(:,j)^T * E \nonumber \\
(j = 0,1,2,\cdots,n)\nonumber
\end{eqnarray}

In summary, the matrix form of updating $\theta$ can be expressed below:
$$\Theta = \Theta - \alpha*X^T*(h_\theta(x)-y)$$
Thus will pave a convenient way to express in MATLAB. Also, we can express cost function and gradient in matrix form.

\end{document}
